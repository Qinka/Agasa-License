% Agasa-License
% An open source, named FROM 阿笠博士, a role in 名探偵コナン
% 阿笠-许可证
% 一个开源协议,命名由来是 日本漫画《名侦探柯南》中的的角色 阿笠博士

%%%%%%%%%%%%%%%%%%%%%%%%%%%%%%%%%%%%%%%%%%%%%%%%%%%%%%%%%%%%%%%%%%%%
% Copyright (C) Qinka <qinka@live.com>
%
% 版权所有 Qinka <qinka@live.com>
%
% 此许可证 遵循 本许可证
%
% This LICENSE is the license of this LICENSE.
%%%%%%%%%%%%%%%%%%%%%%%%%%%%%%%%%%%%%%%%%%%%%%%%%%%%%%%%%%%%%%%%%%%%

\documentclass[UTF8]{book}
\usepackage{xeCJK}
\setCJKmainfont{WenQuanYi Zen Hei}
\author{Qinka \\ qinka@live.com}
\title{阿笠-许可证\\Agasa-License}

\begin{document}
\maketitle
\tableofcontents

\part{协议}
\chapter{中文内容}
\section{著中权声明}
\begin{quote}
    阿笠 协议本身的倾向是“版权消亡”。“版权的消亡”实质上是对版权归属的“无所谓地”给谁,而不是有没有。
    归于当闭源软件消亡之后,版权的归属倒是可以相对无所谓了。

    所以这里基本上是归结于一种 “著中权” 的概念。著中权的概念出发点是介于 著作权与著佐权。
    一方面希望强调到一种回馈精神,就像 “Copyleft” 那样,要限制 “后续的” 的协议继续继续开源,并符合 Copymiddle 的一种精神。
    但是并不希望想 GPL 这样一类协议继续限制于使用GPL。
    另一方面方面,还是说对于版权的一种尊重。

    同时著中权还有一个含义,就是说,对于一个正常的开源项目,“版权”与“贡献权” 相比后者相对更加重要。
    而一个开源项目中,有一类人是及其重要的,就是主要维护者。主要维护者是使得当前一个项目“繁荣”的核心。
    相对于说一个项目的创始人,版权所有者。当下的每位贡献者与核心贡献者才是整个项目的核心。而贡献这一动态的
    过程,是核心中的一个大核心。

    历史是属于人民的。
\end{quote}
Copymiddle (C) YEARS [WHO]

    
\end{document}